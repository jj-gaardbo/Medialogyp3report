\chapter{Conclusion}
From asking customers at a garden center in Hillerød, and having phone interviews with garden architects from around Denmark, it is clear that the target group would be open to trying virtual reality in relation to their work\todo{Feedback on this one though}. \\

We can conclude that the prototype indeed is more immersive than traditional 2D sketching and 3D viewing of gardens. We can't, however, conclude if our prototype is better than conventional sketching methods, for conveying garden design ideas to customers. From the evaluation we can deduce that using our prototype is faster than traditional 3D modeling, but that it also can't replace 2D sketching for conveying garden ideas.\\

Because of the time constraints for both the usability and immersion test, the participants were not members of the target group, and the data produced is therefore only indications for what it might have been. To have more conclusive data, one would get in contact with more landscape architects willing to test the product.\\

From the usability test, it can be concluded that there were usability issues regarding the software, and the physical box. The software at the time of the test did not have rotation functionality\todo{Is this introduced in the test?} and the acrylic plate on top of the box was bigger than what the camera could record, hence resulting in less than desirable usability\todo{Write better}.

From the immersion test we can conclude that virtual reality improves immersion, spatial detailing and understanding of the conceptual design compared to a 2D sketch and a fly through 3D rendering.\\

In regards to our final problem statement:\\
\begin{quote}
	\textit{How can creating a 3D VR environment in a fast and efficient manner, using fiducial markers on a physical implementation, help garden architects give their customers more insight into what it would be like to be in the garden during their design process at the customers garden?}\\
\end{quote}
it isn't possible to conclude whether or not our prototype actually helps garden architects. From the participants acting like garden architects, we can however conclude that it did make the participants understand the conceptual design of the garden faster than 2D sketching and 3D viewing. The participants did respond positively to the prototype, and thought it would be useful for garden architects to use with their customers.\\

\begin{comment}
Can't conclude if clients would benefit from this\\\\
Can't conclude if real garden architects would benefit from this\\\\
Can   conclude that virtual reality helps immersion and spatial detailing\\\\
Can conclude that virtual reality is faster than normal sketching and 3D flyby\\\\
Can't really conclude that garden architect target group would benefit\\\\
But can conclude that participants role playing as garden architects would find it useful\\\\
Can conclude that virtual reality helped participants acting as garden architect clients to better understand the garden design.\\\\
Can't conclude if physical tokens help process or not\\\\
Can't conclude if product makes betterer 3D virtual environment\\\\
Can deduce that it is faster than traditional 3D modeling\\\\
Maybe conclude virtual reality is cool and might help showing normal 3D environments instead of our token based version\\\\
Can't conclude if included plants was enough to diversify garden for participants\\\\
\end{comment}
