\chapter{Design}

\section{The Physical Prototype}

The prototype consists of a lightweight box which can be easily assembled and disassembled, of which four sides are laser cut MDF and the top is a square piece of plexiglass. The bottom is open. A number of tokens may be moved around the surface of the box. The tokens are circular and 3D printed, with a colored marker on the bottom and a picture of the token's virtual counterpart on the top. A camera is placed at the bottom of box, looking up. A light is placed next to the camera, shining on the side of the box to create a diffused light. 
The setup is connected to an HTC Vive with no modifications.

\subsection{The box}

The construction of the box consists of four identical sides measuring x by x by x mm. \todo{get exact measurements}. MDF, or medium-density fibreboard, was chosen for the box as it is lightweight, sturdy, and easy to cut with a laser cutter. It is important that the prototype is lightweight as it must be transported to various locations for testing. The precise laser cutting of the sides means that the sides can be made with the interlocking parts that connect firmly without the need for mechanical locking parts. The main concern was ensuring the pieces form a tight fit. Since material is removed during the laser cutting process, the design had to be purposely made too tight. The test cut worked well at first, but the fit became too loose after being put together and taken apart a few times. A tighter fit was then designed for the final cut of the prototype elements. If the prototype experiences the same issue, we must attach hasps or a similar solution to keep the sides from coming apart. 


The box is made to be tall enough that the camera can capture all of the plexiglass surface in its field of view, which we determined was \todo{height of box}.
The construction was designed to be rectangular as this is the shape of the average garden. %citation needed

The top of plexiglass is needed so that there is a place to put the tokens, which the camera can see. Acrylic was considered, but was too expensive for this project. 


\subsection{The markers}

The physical construction of the markers was not a cause of great concern. They simply needed to be circular and \todo{marker size?} mm in diameter. We chose 3D printing for the tokens for consistency's sake, and for the easy of construction.

\subsection{The headset}
\subsection{The camera}

\section{The Virtual Environment}

\subsection{The UI}
\subsection{The models}
