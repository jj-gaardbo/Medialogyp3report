\chapter{Introduction}
	Designing a garden can be a time demanding affair. People with a newly built house or garden might hiring a landscape architect to take on the design process. As technology evolves, new mediums to demonstrate great garden designs to a client get introduced\cite{landscapeArchitectureDigiTech}. First it was 3D scenes, conveying the design of a garden through a video tour. Now, the next big leap in visualization technology is Virtual Reality\cite{VRS}, and with it comes the ability to show off garden designs to clients in an entirely new way.\\
	
	Some garden architects might find it hard to fully incorporate three dimensional designs into their workflow. The process is time consuming and the software can be difficult to learn, hence only a few actually make use of it. Instead they stick with traditional sketching by hand to communicate their idea to the customers, who risk not being able to fully and accurately envision how the garden will look when it has been completed.\\
	
	In recent years more and more products, such as VR Gardens and Invita Kitchen VR, have been developed that take the three dimensional aspect to another level by incorporating immersive Virtual Reality(VR) into architecture and interior design, which allows a user, wearing a Head Mounted Display(HMD), to step into, and experience, a virtual representation of their future kitchen or garden.\\
	
	For the purposes of this project, the focus will be on the design process of a landscape architect and the quality and speed of the concept visualization during that process. A prototype of an alternative tool for landscape architects which utilizes immersive VR will be presented, evaluated and discussed. A comparison of the prototype and traditional visualization methods, 2D sketches and 3D renderings, will be made with the focal point being the percieved immersiveness.
	
	
	\section{Initial problem statement}
	How can the garden design process be improved by putting the user in a 3D VR environment created from a simple sketch using image processing?
	