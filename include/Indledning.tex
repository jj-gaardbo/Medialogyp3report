\chapter{Introduction}
	
	The design process of landscape architects has been heavily influenced by modern digital technology over the past 50 years, since Harvard's Laboratory of Computer Graphics was established.\cite{landscapeArchitectureDigiTech} In spite of landscape architects being first-movers in exploring digital applications for spatial analysis, many landscape architects are struggling with conceptualizing digital technology as a creative medium\cite{landscapeArchitectureDigiTech}. \\
	\\
	A survey on members of ASLA (American Society of Landscape Architects)\cite{surveySketchVSDigital}, shows that a 46\% of landscape architects prefer sketching by hand during the design process instead of using digital tools, 31\% prefer using a computer and the remaining 23\% use the computer for efficiency and the hand for creativity. Based on the statements given by the respondents the survey concludes that; \textit{"computers are not intuitive and design is intuitive"}\cite{landscapeArchitectureDigiTech}\cite{surveySketchVSDigital}. Although, computer software enables visualization in 3D which is an effective way of communicating large amounts of complex data to a wide non-expert audience, through visual cues that are more intuitive than those of a 2D-sketch, it can be time-consuming to create 3D visualizations\cite{landscapeVisual}
	
	\section{Initial problem statement}
	How can the garden design process be improved by putting the user in a 3D VR environment created from a simple sketch using image processing?
	