%This file should include Discussion, Future works and Conclusion
\chapter{Discussion}
When evaluating the data acquired during the earlier chapter, it is important to account for two things: validity and reliability.
\section*{Validity}
	Validity is defined as:\\
	\begin{quote}
		\textit{Validity is concerned with whether the evaluation method measures what it is intended to measure}\cite{interactionDesign}.\\
	\end{quote}
	What this roughly means, is how much the data can be trusted, as in if the there can be potential variations that might invalidate the data. Some potential causes for and against the validity of our two tests could be:\\
	\begin{enumerate}
		\item The tests both took place in our group room, where there could have been noise, and distractions from outside people, that could have made the participants less focused on the test at hand.\\
		
		\item Most of, if not all, of the participants that were found utilizing convenience sampling, were students AAU CPH, and as such aren't very likely representative of how the actual target group would have behaved during the test.
		
	\end{enumerate}
\section*{Reliability}
	Reliability is defined as:\\
	\begin{quote}
		\textit{The reliability or consistency of a method is how well it produces the same results on separate occasions under the same circumstances}\cite{interactionDesign}.\\
	\end{quote}
	Which means how much you can rely on the tests producing the same results, under the same circumstances, if the test was run on separate .\\
	\begin{enumerate}
		\item For the usability test, we made a plan for how the test should be set up, as seen in \autoref{fig:test1}. This was to make it so that every instance of the test ran as close to identically as possible.\\
		
		\item For the immersion test, there's was no such plan, and due to the test being split up into 3 separate parts, it was semi-structured at best. Due to the semi-structured setup of the test, there were also no minimum or maximum time each participant spent on each part, they simply spent the time they each thought was adequate.\\
		
		\item The usability test followed a manuscript, to make sure there was as little inconsistency as possible.\\
		
		\item The immersion test did not have any manuscript, but merely some rough and loose guidelines to follow; explaining about the three parts of the test, and that the participant should try to visualize being in the garden.\\
	\end{enumerate}
	
\section {Future Works}


\section {Conclusion}
In order to get a comfortable and aesthetically pleasing garden around the house, one could get in touch with a landscape architect, although if a simple 2D sketch is not a sufficient visualization of their design there are most likely going to be additional expenses involved, and possibly also a longer waiting period.

The landscape architects usually include their clients in the design process and therefore it would be relevant to consider these clients as a secondary target group. Based on the client related questions it is clear that the target groups primary customers range from around 25 to 60 years. There are no distinctive design elements that are heavily used. The trends are that the gardens should be maintenance-free and have plants that are able to survive the Danish climate due to shifting weather through the year. The architects customer can sometimes have a hard time understanding the three-dimensional space of their future garden when it is only presented on a sketch. Even though some landscape architects use 3D visualization it has to be an easy and efficient process in order to keep the cost down. Otherwise the customers will probably do without it.

Most of the questionnaire respondents were familiar with VR and felt comfortable using it. The group that expressed most comfort with it were aged 36-45, which is within the age group defined by the experts in section\autoref{sec:expertInterviews}
The garden sizes vary quite a bit although many of the respondents' gardens were larger than 800m² and the second largest group were in between 500-800m². This indicates that the ability to scale the garden would be useful. If the project ends up being limited in the number of models included, it would make sense to at least include the common garden plants shown in \autoref{fig:plantlist}.