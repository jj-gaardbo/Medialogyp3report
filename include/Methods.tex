\chapter{Methods}
\label{chp:methods}
%Strategy chosen in order to answer the final problem statement.
The goal of the project was to create a final design that could be used to answer the final problem statement. This section aims to explain the strategy and methods employed in order to create that design and answer the final problem statement.\\

Ideally, we would have taken a user centered design approach, meaning that end users would be involved early in the design process, and the design would be based around their needs and wants. However, garden architects proved to be an illusive target group. There were not very many of them and they were spread over large areas. None of the group members were in possession of a car, so transport to the target group was unfeasible in the allowed time frame. In lieu of end users, we were forced into what Preece, Rodgers, and Sharp defines as "Genius Design"\cite[p.~346]{interactionDesign}, writing that.\\
	\begin{quote}
	\textit{In this approach the users’ role is to validate ideas generated by the designer, and users are not involved during the design process itself. Saffer points out that this is not necessarily by choice, but may be due to limited or no resources for user involvement.}\\
	\end{quote}

This is still a design with the end user in mind, however the designers base the design requirements on knowledge about the users, acquired through interviews and research, and their own experience in the field. In accordance the final problem statement should be answered by testing on the target group.\\

Based on research and target group interviews we established the design requirements. In accordance with genius design, the prototype were to be evaluated on garden architects. From the final problem statement we had two things to evaluate. Usability in regards to the usability goals of being fast and efficient for the architects, and immersiveness to aid in spacial understanding for the customers.

\section{Expert usability evaluation}
A good way to test usability, would be to have experts use the prototype in a real world context in a natural environment, using a think-out-loud technique. They would be recorded with a video camera during the test, and usability observations could be made from the recordings. After the using the prototype they would fill out the System Usability Scale questionnaire\footnote{Usability. gov SUS : \url{https://www.usability.gov/how-to-and-tools/methods/system-usability-scale.html}} that would be used as a basis for semi-structured interviews after. The interview would be analyzed using traditional coding, and statement trends would be identified. This was chosen based on a low amount of participants.

\section{Customer immersion}
Customer immersion would be measured quantitatively, as large sample sizes were needed to establish the validity amongst the target group population. The participants would try two tools traditionally used by garden architects to communicate their design idea, paper and a 3D fly through, as well as our prototype. They would try each of them, in every possible sequence combination. After trying each tool, they would rate the tool's immersion using a questionnaire. The results would be found by analyzing this questionnaire.