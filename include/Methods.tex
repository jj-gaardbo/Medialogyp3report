\chapter{Methods}
%Strategy chosen in order to answer the final problem statement.
This section aims to explain the strategy and methods employed in order to answer the final problem statement. 

\section{Approach to Design and Implementation}

%Why did we do tokens?
Our initial idea was to allow for the user to produce a simple sketch of a garden for the camera to recognize eg. colours, shapes, symbols etc. This seemed to be the ideal way of doing the prototype as it would allow creative freedom for the user in what by definition is a creative process. However, the issue with sketching ended being consistency. We still wanted the structure to be simple, but the elements of the prototype to be controlled for consistent recognition. So the use of tokens or fiducial markers came to be the essential solution to achieving the consistency, by having recognizable tokens that should be manipulated within the physical prototype.\\


%Talk about demo day, vr and such.
In order to know how to approach the design and implementation phases, we knew we had to get a grasp of how our prototype should work. Primarily we had to test initial ideas and produce an initial prototype for testing at the mid term demo day on AAU CPH campus to get results. This "test" was essentially fueled by convenience sampled test subjects, who most likely didn't have their own garden, or a need to design one, as they were exclusively students at AAU CPH. \\

%Not very scientific.
With the initial idea we did not have a specific scientific approach to create an initial semi-functioning prototype. Preparing for the mid term demo day on AAU CPH campus, meant doing rapid small prototypes of the image processing implementation leading up to the event.\\


%Talk about small prototypes to reach goal.
During the project duration, many rapid and small implementation prototypes should be made, to be absolutely certain that the project develops alongside the implementation of the problem solution.
To reach our goal of having a functioning and testable prototype, it was needed to know how the image processing aspect should function, in order for it to be most efficient and reduce workload simultaneously. Furthermore, these small prototypes will be used for testing along the way, to help structure the concept of a final prototype.\\

%No interface, not much in the design apartment, no real low fidelity prototype.
During early implementation it became apparent, that our prototype would not feature an actual visual interface in virtual reality, except the world itself, and the physical surface the placeholder tokens were placed on.
%Why no interface inside vr?
Having the tokens spread on a transparent surface representing the garden design, these physical objects became the interface between the user and the virtual reality, which means that our interface design is to be done completely outside of the digital implementation, focusing entirely on the physical representation of the garden.

%Possible directions for the design?


\section{Initial Research}
\subsection{Garden Centre Survey}

In order to learn more about the people we were designing for, we went where we were sure to find homeowners with their own garden and an interest in keeping that garden nice. Plantorama is one such place, a chain of garden centres with three locations in the Greater Copenhagen area alone. Our location of choice was the Hillerød centre, the largest roofed garden centre in the Nordic countries. 

There, we went to gather responses for a survey. 
The purpose of the survey was to learn the following:

\begin{itemize}
	\item Who we should expect to be uncomfortable with VR.
	\item Whether or not our product would have a measurable impact if set up in a garden centre.
	\item What sizes our virtual garden must be able to emulate.
	\item Which plants to create as 3D models.
	\item Customers' purchasing ability
	\item Whether the customers' focus are on in- or outdoor plants
	\item Whether or not our proposed product could alleviate problems with choosing plants to buy
\end{itemize}
A survey was created using google forms which would provide insight to answer these questions, and printed in 37 copies. The final survey and results are included in the appendices. The surveys were handed out to customers at Plantorama Hillerød who had their own garden. Two of the forms were discarded as they revealed the respondent had no garden. %tsk, tsk...

\subsection{Expert Interviews}
We also were interested in discovering how professional garden designers, or landscape architects some prefer to go by, work with their clients to create a plan from start to finish. We were particularly curious about their relationship with technology, and to what degree they used 3D visualizations in their work and presentation for the client. 

In \autoref{sec:expertInterviews} it is described how the interviews with these experts were conducted over the phone and how they were recorded with permission and the transcribed versions are available in the appendices. 

\section{User Testing}

\subsection{Expert test}
TODO
\subsection{Usability test}
TODO
