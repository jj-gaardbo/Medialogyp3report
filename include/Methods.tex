\chapter{Methods}
%Strategy chosen in order to answer the final problem statement.
This section aims to explain the strategy and methods employed in order to answer the final problem statement. 

\section{Approach to Design and Implementation}

%Why did we do tokens?
The initial idea was for a camera to recognize colours, shapes, sizes, and symbols on a simple, user created sketch of a garden. This seemed to be the ideal prototype as it would give the user a lot of creative freedom in the creative process. However, the issue with sketching ended up being consistency. We wanted the structure to be simple, and the elements of the prototype to be controlled for consistent recognition. The use of tokens or fiducial markers was determined to be the best solution for achieving consistency. The tokens would be physical objects manipulated within the physical prototype.\\

%Talk about demo day, vr and such.
In order to be more informed going through the design and implementation phases, we had to get a grasp of how our prototype should work. We had ideas and an initial prototype to test at the midterm "Demo Day" on AAU CPH campus in the hopes of getting useful feedback. This informal test was conducted using convenience sampled test subjects, who had no garden of their own, nor any real need to design one, as they were exclusively students at AAU CPH. At this point in time, the prototype allowed only for the placement of either trees or bushes.\\

%Not very scientific.
With the initial idea we did not have a specific, scientific approach to create an initial, mostly functional prototype. Preparing for the Demo Day involved rapid prototyping of the image processing algorithm and Unity integration leading up to the event.\\

%Talk about small prototypes to reach goal.
During the project, many rapid and small implementation prototypes should be made to ensure steady progress in the project.
To reach our goal of having a functioning and testable prototype, an idea of how the image processing functionality should work was needed. The prototypes will be tested as they are completed, each informing the development of the next, ultimately culminating in a final, fully functional prototype.\\

%No interface, not much in the design department, no real low fidelity prototype.
During the early implementation phase it became apparent that our prototype would not feature a digital interface, letting the user interact only with the virtual world itself when putting on the VR headset. The physical surface and the tokens placed on it would act as the only way to manipulate the world.

%Why no interface inside vr?
Using a physical interface makes the process of manipulating the virtual world much more approachable to those not familiar with VR. It also encourages creativity more so than a digital system would. \todo{Find a source for this}


%Possible directions for the design?


\section{Initial Research}
\subsection{Garden Centre Survey}

In order to learn more about the people we were designing for, we went where we were sure to find homeowners with their own garden and an interest in keeping that garden nice. Plantorama is one such place, a chain of garden centres with three locations in the Greater Copenhagen area alone. Our location of choice was the Hillerød centre, the largest roofed garden centre in the Nordic countries. 

There, we went to gather responses for a survey. 
The purpose of the survey was to learn the following:

\begin{itemize}
	\item Who we should expect to be uncomfortable with VR.
	\item Whether or not our product would have a measurable impact if set up in a garden centre.
	\item What sizes our virtual garden must be able to emulate.
	\item Which plants to create as 3D models.
	\item Customers' purchasing ability
	\item Whether the customers' focus are on in- or outdoor plants
	\item Whether or not our proposed product could alleviate problems with choosing plants to buy
\end{itemize}
A survey was created using google forms which would provide insight to answer these questions, and printed in 37 copies. The final survey and results are included in the appendices. The surveys were handed out to customers at Plantorama Hillerød who had their own garden. Two of the forms were discarded as they revealed the respondent had no garden. %tsk, tsk...

\subsection{Expert Interviews}
We also were interested in discovering how professional garden designers, or landscape architects some prefer to go by, work with their clients to create a plan from start to finish. We were particularly curious about their relationship with technology, and to what degree they used 3D visualizations in their work and presentation for the client. 

In \autoref{sec:expertInterviews} it is described how the interviews with these experts were conducted over the phone and how they were recorded with permission. The transcribed versions are available in the appendices. 

\section{User Testing}

\subsection{Expert test}
TODO
\subsection{Usability test}
TODO
