\chapter{Methods}
%Strategy chosen in order to answer the final problem statement.
This section aims to explain the strategy and methods employed in order to answer the final problem statement. 

\section{Approach to Design and Implementation}

%Why did we do tokens?
The initial idea was for a camera to recognize colours, shapes, sizes, and symbols on a simple, user created sketch of a garden. This seemed to be the ideal prototype as it would give the user a lot of creative freedom in the creative process. However, the issue with sketching ended up being consistency. We wanted the structure to be simple, and the elements of the prototype to be controlled for consistent recognition. The use of tokens or fiducial markers was determined to be the best solution for achieving consistency. The tokens would be physical objects manipulated within the physical prototype.\\

%Talk about demo day, vr and such.
In order to be more informed going through the design and implementation phases, we had to get a grasp of how our prototype should work. We had ideas and an initial prototype to test at the midterm "Demo Day" on AAU CPH campus in the hopes of getting useful feedback. This informal test was conducted using convenience sampled test subjects, who had no garden of their own, nor any real need to design one, as they were exclusively students at AAU CPH. At this point in time, the prototype allowed only for the placement of either trees or bushes.\\

%Not very scientific.
With the initial idea we did not have a specific, scientific approach to create an initial, mostly functional prototype. Preparing for the Demo Day involved rapid prototyping of the image processing algorithm and Unity integration leading up to the event.\\

%Talk about small prototypes to reach goal.
During the project, many rapid and small implementation prototypes should be made to ensure steady progress in the project.
To reach our goal of having a functioning and testable prototype, an idea of how the image processing functionality should work was needed. The prototypes will be tested as they are completed, each informing the development of the next, ultimately culminating in a final, fully functional prototype.\\

%No interface, not much in the design department, no real low fidelity prototype.
During the early implementation phase it became apparent that our prototype would not feature a digital interface, letting the user interact only with the virtual world itself when putting on the VR headset. The physical surface and the tokens placed on it would act as the only way to manipulate the world.

%Why no interface inside vr?
Using a physical interface makes the process of manipulating the virtual world much more approachable to those not familiar with VR. It also encourages creativity more so than a digital system would. \todo{Find a source for this}


%Possible directions for the design?


\section{Initial Research}
\subsection{Garden Centre Survey}

In order to learn more about the people we were designing for, we went where we were sure to find homeowners with their own garden and an interest in keeping that garden nice. Plantorama is one such place, a chain of garden centres with three locations in the Greater Copenhagen area alone. Our location of choice was the Hillerød centre, the largest roofed garden centre in the Nordic countries. 

There, we went to gather responses for a survey. 
The purpose of the survey was to learn the following:

\begin{itemize}
	\item Who we should expect to be uncomfortable with VR.
	\item Whether or not our product would have a measurable impact if set up in a garden centre.
	\item What sizes our virtual garden must be able to emulate.
	\item Which plants to create as 3D models.
	\item Customers' purchasing ability
	\item Whether the customers' focus are on in- or outdoor plants
	\item Whether or not our proposed product could alleviate problems with choosing plants to buy
\end{itemize}
A survey was created using google forms which would provide insight to answer these questions, and printed in 37 copies. The final survey and results are included in the appendices. The surveys were handed out to customers at Plantorama Hillerød who had their own garden. Two of the forms were discarded as they revealed the respondent had no garden. %tsk, tsk...

\subsection{Expert Interviews}
We also were interested in discovering how professional garden designers, or landscape architects some prefer to go by, work with their clients to create a plan from start to finish. We were particularly curious about their relationship with technology, and to what degree they used 3D visualizations in their work and presentation for the client. 

In \autoref{sec:expertInterviews} it is described how the interviews with these experts were conducted over the phone and how they were recorded with permission. The transcribed versions are available in the appendices. 

\section{User Testing}
Warning. The content you're about to read (until the end of methods) is a first pass of a first draft. It's basically a pile of shit that's been eaten, digested, and shat out again. It's the only way I can write though, or I'll be stuck refining one paragraph for a whole day. You shouldn't even be reading this, that's how unfinished it is. \textbackslash Nicolai \\\\ %only god can judge me
Why are we testing our own product, anyway? Let's test electric tasers. That honestly sounds a lot more fun. But sure, we can test our product... it's fine. Ok. Alright.
\subsection{Expert test}
We need to get an expert's opinion on how usable our product is. How well is it designed? Is it easy to use? Will a client be comfortable using it? Does it take up too much space? Is it too complicated to set up in its current state? Can it run on a normal computer? Should we be concerned with that at all? It could be ignored because ideally we can just put the program on a vive focus or oculus go, but then we would need to create a way for the webcam to transmit data to the headset. Presumably the headsets will have a wifi card. Bluetooth? Probably not. But no specs are available for any of the headsets which are not out yet. So it's a bit pointless to start speculating about how to implement it. For all we know they're using smoke signals.
%I can't believe I'm sharing a kebab with the most beautiful girl I've ever seen eating a kebab
\subsubsection{Type of observation}
What type of observation? Natural or artifical environment, participant or non-participant observation?
Human or mechanical observation? Direct or indirect?

While it could provide useful insight to test in a natural environment, the natural environment is in this case the client's home. And acquiring a home with a garden seems excessive when we might as well conduct the test in the expert's office or at our own school facilities. So it will probably be an artificial or make-shift artificial environment. To get proper knowledge of how shit is gonna go down, we need to place a group member as a participant in the test. The group member will act as the client looking for a new design of their garden. The expert will carry out their job as usual, except they will now utilize our product to help the "client" visualize the garden design.\\
Mechanical observations will be performed: We need a log of frames-per-second to see how effeciently the program performs. Otherwise we will have human people taking human notes on regular paper. Just kind of observing what is going on.

\subsubsection{Strategy for observation}
What strategy for conducting observation? What's the research question and will this test answer it? Outline what will observed and why. Consider time of day / day of week, position of cameras/observers and participants. How will you be taking notes? What in particular will you be paying attention to? What precautions will you take to prevent bias? How will the observation, the participant, the observer, and the environment impact the test? 
How will the data collected be analyzed?
%Binary solo! 00011001 11000010 11110111 01110001 11001100 11111111

Seriously, what's the question again? "How to improve the garden design process?" Or is it "How to help visualize the newly designed garden for the client"? Anyway, the observations will show any problems because we're doing the actual thing that the product will be used for. But if we knew what questions to ask, it would be better. The expert will come with suggestions of their own, probably like "You're missing plant x, y and z" And we'll go "Well that's not really important right now". 

So one thing that will prove difficult in this test is that the roles as far as virtual reality goes are reversed. The idea is that the landscape architect brings the vr headset to the client and instructs them how to use it. But in the test, it is us, the "clients", who will be instructing the expert in how to instruct us to use it. It all gets very confusing. So maybe we won't go for accuracy as far as that part is concerned, but it's important we're aware that the test won't emulate a normal expert-client meeting perfectly. 

We will got through different expert-client scenarios. One where the client wants just a small, simple garden, a more complicated one, and a large garden with a ton of objects. It's important to find out how the product performs in different situations. 

The environment we're testing in shouldn't matter too much as long as the expert is willing to "roleplay" and act as if we're a real client - at least to an extent where the product will be put through everything it would in a real scenario. 

%I'm actually disappointed latex doesn't support emojis
 
\subsubsection{Interview method}
What interview method? (Expert interview / in depth?)




\subsection{Usability test}
This could be coupled with the expert landscape architect tests and/or be carried out on "normal" people around campus or alternatively in a garden center. Maybe all of the above. Need to test both the completely basic "can this product be used without being endlessly frustrating" aspect, and the "does this product help answer the problem statement" thing.
\subsubsection{Type of observation}
What type of observation? Natural or artifical environment, participant or non-participant observation?
Human or mechanical observation? Direct or indirect?

Definitely need a mechanical framecounter for this one. \\ %because counting frames manually is really difficult
It would be advantageous to test users in both "designer" and "client" roles. It might be a good idea to have to test subjects doing one role each at the same time, as to minimize our (the observers) influence over the participants. If we participate in the test we might affect the actions of the participant. The actions performed by two test participants playing off each other may not necessarily represent realistic scenarios, but it's certainly possible, and it would be unfortunate to miss out on that juicy, juicy data. This would require the participants to be given some direction in advance, so they don't just resort to "playing", which is a risk since VR is a fun technology. Could utilize think-out-loud technique on the participants, or perhaps interview them afterwards.
\subsubsection{Strategy for observation}
What strategy for conducting observation? What's the research question and will this test answer it? Outline what will observed and why. Consider time of day / day of week, position of cameras/observers and participants. How will you be taking notes? What in particular will you be paying attention to? What precautions will you take to prevent bias? How will the observation, the participant, the observer, and the environment impact the test? 
How will the data collected be analyzed?
\subsubsection{Interview method}
We could do an interview after or just observe during test. Hmmm...